\documentclass[../main.tex]{subfiles}

 
\begin{document}

In the written report you summarize your work. The report will be archived and, if you wish, published. To create uniformity for the technical masters, we have from the spring semester of 2017 the requirement that the report should follow the instructions in this document. It facilitates feedback between students and teachers when everyone involved knows what is expected of the report. Moreover, you will get a good exercise in following the instructions for the drafting of a document, which you will have advantage of no matter where you choose to work in the future.

This document is thus a template, and contains instructions for the report for Master programs in Bioinformatics and Molecular Biotechnology, at Uppsala University. This document supplements the course plans, the faculty guidelines for degree projects (TEKNAT 2012, IBG 2016a), specific instructions for the master's programs, and Presenting Science, IBG:s instructions on scientific writing (Rydin et al. 2014).

You can start writing your report by replacing the text in the template with your own. A Microsoft Word version is available for downloading from the course page if it is the PDF version you are reading. The template uses the styles that are pre-defined with names that begin with X. The names of the styles are shown in bold in this document. The styles for example control spaces between paragraphs. So, never use additional blank lines. If you use other software, follow the appearance that the formatted template provides (see Appendix A).

The work is presented in the form of a report written in English (or Swedish) with abstract in English. If the report is written in English, use British English spelling and language conventions. The report has three parts, the introductory part with information and summaries of the report, the main body in which you present your project, and a final section with information about references, and possibly appendices.

At IBG we follow the instructions in Presenting Science. The booklet has instructions for scientific writing within the biological field. Different disciplines, however, have different traditions when it comes to the outline, and many degree projects are performed in areas not covered by Presenting Science. Therefore, consult your supervisor or subject reader about the appropriate disposition for the type of work you have done. However, you must always follow the instructions for referencing, tables and figures according to Presenting Science and formats in this document that control the layout.


\subsection{Some subsection}

Here you explain more things

\subsection{Some subsection v2}

Here you explain even more things

\subsubsection{Here you are being too presumptuous my friend}

Here you try to be super clever but is probably not necessary to talk about that
\end{document}
